\documentclass[a4paper,11pt]{article}
\usepackage{fullpage}
\usepackage[colorlinks=true]{hyperref}
\usepackage{amssymb}
\usepackage{amsmath}
\usepackage{fancyhdr}
\usepackage{verbatim}
\usepackage{lmodern}
\usepackage{listings}
\usepackage{graphicx}
\usepackage{booktabs}
\usepackage{caption}
\usepackage{color}
%\usepackage{bibtex}

\setlength{\headheight}{30pt}
\setlength{\headsep}{0.2in}

\definecolor{gray}{rgb}{0.5, 0.5, 0.5}

\lstset{
    language=Python,
    basicstyle=\footnotesize,
    numbers=left,
    numberstyle=\scriptsize\color{gray},
    stepnumber=1,
    numbersep=5pt,
    backgroundcolor=\color{white},
    showspaces=false,
    showstringspaces=false,
    rulecolor=\black,
    tabsize=4,
    captionpos=t,
    breaklines=true,
    keywordstyle=\color{black},
    commentstyle=\color{black},
    stringstyle=\color{black}
}

\title{OpenCUDA+MPI\\
       A Framework for Heterogeneous GP-GPU Distribute Computing}
\date{\today}
\author{Kenny Ballou\\
        College of Engineering -- Department of Computer Science\\
        Boise State University\\
        Alark Joshi, Ph. D}

\lhead{\fancyplain{}{Student Research Initiative 2013\\
                     Progress Report}}
\pagestyle{empty}

\begin{document}
\begin{titlepage}
\begin{center}

\textsc{\LARGE OpenCUDA+MPI}\\[1.5cm]
\textsc{\Large A Framework for Heterogeneous
               GP-GPU Distributed Computing}\\[1.5cm]

Kenny Ballou\\
College of Engineering -- Department of Computer Science\\
Boise State University\\
Alark Joshi, Ph.D\\
\vfill
\end{center}
\end{titlepage}

\nocite{*}
\thispagestyle{fancy}
\abstract{The introduction and rise of General Purpose Graphics Computing has
significantly impacted parallel and high-performance computing. Even worse, it
has brought about even more challenges when it comes to distributed computing
(with GPU's). Current solutions target specifics: specific hardware, specific
network topology, a specific level of sameness. What if we could ignore
specifics? What if we could write a general algorithm to solve a problem and
have a job scheduler do the rest? That is the goal of OpenCUDA+MPI. To achieve
this goal, we have written a framework that allows a developer/ data scientist
to write a general algorithm/ solution without the overhead of worrying about
the specifics of the cluster it will run against. We have found, [nothing yet]
and, therefore, can conclude [also nothing].}\\

\textbf{Keywords:} Parallel Computing, Distributed Computing, General Purpose
Graphics Processing, High-Performance Computing, Scientific Computing
\section{Introduction}

Graphics processing units (GPU) have significantly altered the way
high-performance computing tasks can be performed today. A major challenge
remains in being able to seamlessly intergrate multiple workstations to further
parallelize the computational tasks. Current approaches provide the ability to
parallelize tasks, but they are less focused on optimally utillizing the varied
capabilities of heterogeneous graphics cards in the cluster of workstations.\\

\subsection{Outline}

\subsection{What Others Have Done}

% vim: syntax=tex:
\section{Methodology}

Our framework will be developed using a number of tools and will also build
upon a few already developed and mature software libraries. Overall, we plan to
use the \texttt{Python} programming language, \gls{mpi} for cross process
communication, and \gls{cuda} for graphics computing.

\subsection{Python}

We will use \texttt{Python} programming language because of its ease of
development and plethora of existing libraries such as \texttt{mpi4py} and
\texttt{pyCUDA}. \texttt{Python} also adds some advantages when it comes to
usability when needing more performance. For example, if we discover a need
for certain aspects of the project to be tuned or otherwise run faster, we can
easily switch to C or C++ and write sections of the program in a (more
performant) native language.

\subsection{MPI}

Currently our cluster consists of 16 computers provided to our research lab by
the Computer Science department of Boise State University. We will use
\gls{mpi} because it has established itself as the de-facto interface for cross
process communication \cite{website:MPI-Tutorial}. To allow for interfacing
with \texttt{Python} \cite{website:mpi-4-python}, we will use \texttt{mpi4py}
because of its maturity and implementation completeness.

\subsection{CUDA}

We will be using \gls{cuda}, and in particular, \texttt{pyCUDA}, because of
existing knowledge of the library/ framework and also because it is a well
established library for \gls{gpu} computing.

\subsection{\texttt{OpenCUDA+MPI}}

Our frameworks' goal is to make accessible the power of \gls{cluster} computing
without necessarily knowing in-depth the complexities of inter-computer related
communication. Further, in doing so, we would like to expose functionality that
may not be available even in more established \gls{cluster} libraries and
frameworks. Namely, facilities for debugging and profiling.

\subsection{Testing}

To evaluate the efficacy and accuracy of our framework several tests will be
executed. As previously mentioned one of the first algorithms to be tested will
be the problem of vessel extraction. Accurately extracting vascular structures
from a Computerized Tomographic angiography--also called CT angiography
scans--is important for creating oncologic surgery planning tools as well as
medical visualization applications \cite{erdt2008automatic}.  Currently, we use
a single \gls{gpu} to extract vascular structures from a CT angiography scan,
which is computationally intensive. The following test programs will be
developed as time allows:

\begin{itemize}
    \item N-Body Simulation
    \item Prime Number Searching
\end{itemize}

Every test program will be evaluated in three categories: \gls{cpu}-only,
\gls{cuda}-only, \gls{cuda} + \gls{mpi} and \texttt{OpenCUDA+MPI}. Having all
of the prior solutions that do \emph{not} use the framework provides a baseline
time comparison and provide immense insights into the pains and difficulties we
are actually attempting to solve.

%\begin{block}{\Large Results}
We were able to show that distributing work over a cluster can
significanly decrease the amount of time required for a problem. For example,
processing the N-Body problem for 200 thousand bodies on a single CPU required
$114349$ seconds (about 31 hours), as compared to a single GPU which required
$52$ seconds and distributing over 16 GPU's only took $21$ seconds. For more
comparisons, please see the table below. [Currently, a comparison between the
distributed code with and without the framework is not available.]\hfill{}\\{}

Although biased, it was interesting to notice the development time difference
between writing the distributed code without the framework as compared to with
the framework. Namely, without the framework, it required several weeks,
whereas, with the framework, it required less than a week.
\end{block}
%\begin{block}{}
%\begin{table}[h]
%\begin{tabular}{llcc}
%\toprule{}
%\textbf{Size} & \textbf{Method} & \textbf{Elapsed (seconds)} & $\sigma{}$ \\
%\midrule{}
%2000            & CPU          & 30.76    & 0.72   \\
%\midrule{}
                %& GPU          & 14.14    & 0.76   \\
%\midrule
                %& CUDA+MPI     & 819.67   & 807.29 \\
%\midrule
%20,000          & CPU          & 2393     & 10.30  \\
%\midrule{}
                %& GPU          & 26.82    & 0.062  \\
%\midrule{}
                %& CUDA+MPI     & 12.50    & 11.34  \\
%\midrule{}
%200,000         & CPU          & 114349   & 411.44 \\
%\midrule{}
                %& GPU          & 52.86    & 0.08   \\
%\midrule{}
                %& CUDA+MPI     & 21.34    & 19.61  \\
%\midrule{}
%2,000,000       & GPU          & 217.60   & 0.82   \\
%\midrule{}
                %& CUDA+MPI     & 97.76    & 20.78  \\
%\midrule{}
%20,000,000      & GPU          & 1509.98  & 2.70   \\
%\midrule{}
                %& CUDA+MPI     & 950.05   & 47.81  \\
%\bottomrule{}
%\end{tabular}
%\end{table}
%\end{block}

%\include{Discussion}
%\include{Conclusions}
\appendix
%\include{Appendices}
% vim: syntax=tex:
\section{Reference}
\renewcommand{\refname}{}
\bibliographystyle{plain}
\bibliography{references}

\end{document}
